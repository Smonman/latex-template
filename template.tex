\documentclass[10pt,titlepage]{article}
\usepackage{template}

%%%
%%% PLACE CUSTOM STYLES HERE
%%%

%%%

\title{Title}
\author{Author}

\begin{document}
\maketitle

\tableofcontents
\newpage

\section{Introduction}
\lipsum[1-1]

\subsection{Subsection}
\lipsum[2-3]

\subsubsection{Subsubsection}
\lipsum[2-3]

\paragraph{Paragraph}
\lipsum[1-2]

\subparagraph{Subparagraph}
\lipsum[1-2]

\section{Itemize and Enumerate}
\subsection{Itemize}
\begin{itemize}
  \item I am an item
  \item I am an item
  \item I am an item
        \begin{itemize}
          \item I am an item
          \item I am an item
          \item I am an item
                \begin{itemize}
                  \item I am an item
                  \item I am an item
                  \item I am an item
                \end{itemize}
        \end{itemize}
\end{itemize}

\subsection{Enumerate}
\begin{enumerate}
  \item I am an item
  \item I am an item
  \item I am an item
        \begin{enumerate}
          \item I am an item
          \item I am an item
          \item I am an item
                \begin{enumerate}
                  \item I am an item
                  \item I am an item
                  \item I am an item
                \end{enumerate}
        \end{enumerate}
\end{enumerate}

\section{Math}

\subsection{Abs and Norm}
Commands \verb|\abs| and \verb|\norm|\footnote{see https://tex.stackexchange.com/a/43009} produce the following:
\begin{table}[!ht]
  \centering
  \begin{tabular}{ccc}
    \toprule
    \thead{command} & \thead{non-starred}       & \thead{starred}            \\
    \midrule
    \verb|\abs|     & \(\abs{\frac{1}{2}x^2}\)  & \(\abs*{\frac{1}{2}x^2}\)  \\
    \verb|\norm|    & \(\norm{\frac{1}{2}x^2}\) & \(\norm*{\frac{1}{2}x^2}\) \\
    \bottomrule
  \end{tabular}
\end{table}

The difference between \textit{starred} and \textit{non-starred} lies in the scaling of the bars.

\subsection{Set}
Command: \verb|\mset| produces the following:
\[\mset{1, 2, 3, 4}\]
\[\mset{a \mid \frac{a}{2} > 5}\]

\subsection{Conditions}
Environment: \verb|\begin{conditions}|\footnote{see https://tex.stackexchange.com/a/95842} can be used for the following:
\\
Boltzmann distribution: state occupation probability of a thermodynamical system within fixed temperature \(T\): \[p(x) = \alpha \cdot e^{-\frac{E(x)}{k \cdot T}}\] where:
\begin{conditions}
  x & \dots & state \\
  \alpha & \dots & degeneracy (= number of states \(x'\) with the same energy as \(x\)) \\
  E(x) & \dots & energy \\
  k & \dots & Bolzmann constant
\end{conditions}
It is possible to have different symbols instead of the dots.

\section{lstlistings}

\subsection{Python}
\begin{lstlisting}[language=python]
import numpy as np
    
def incmatrix(genl1,genl2):
    m = len(genl1)
    n = len(genl2)
    M = None #to become the incidence matrix
    VT = np.zeros((n*m,1), int)  #dummy variable
    
    #compute the bitwise xor matrix
    M1 = bitxormatrix(genl1)
    M2 = np.triu(bitxormatrix(genl2),1) 

    for i in range(m-1):
        for j in range(i+1, m):
            [r,c] = np.where(M2 == M1[i,j])
            for k in range(len(r)):
                VT[(i)*n + r[k]] = 1;
                VT[(i)*n + c[k]] = 1;
                VT[(j)*n + r[k]] = 1;
                VT[(j)*n + c[k]] = 1;
                
                if M is None:
                    M = np.copy(VT)
                else:
                    M = np.concatenate((M, VT), 1)
                
                VT = np.zeros((n*m,1), int)
    
    return M
\end{lstlisting}

\subsection{C++}

\begin{lstlisting}[language=c++]
#include <iostream>
using namespace std;

int main() {
    int n, t1 = 0, t2 = 1, nextTerm = 0;

    cout << "Enter the number of terms: ";
    cin >> n;

    cout << "Fibonacci Series: ";

    for (int i = 1; i <= n; ++i) {
        // Prints the first two terms.
        if(i == 1) {
            cout << t1 << ", ";
            continue;
        }
        if(i == 2) {
            cout << t2 << ", ";
            continue;
        }
        nextTerm = t1 + t2;
        t1 = t2;
        t2 = nextTerm;
        
        cout << nextTerm << ", ";
    }
    return 0;
}
\end{lstlisting}

\subsection{Output}

\begin{lstlisting}[style=lstoutput]
Enter a positive integer: 100
Fibonacci Series: 0, 1, 1, 2, 3, 5, 8, 13, 21, 34, 55, 89, 
\end{lstlisting}

\subsection{Pseudocode}

\begin{lstlisting}[language=pseudo]
function Tree-Search(problem) returns a solution, or failure
  initialize the frontier using the initial state of problem
  loop do
    if the frontier is empty then return failure
    choose a leaf node and remove it from the frontier
    if the node contains a goal state then return the corresponding solution
    expand the chosen node, adding the resulting nodes to the frontier
\end{lstlisting}

\begin{code}[pseudo]{code.pseudo}
  function Tree-Search(problem) returns a solution, or failure
  initialize the frontier using the initial state of problem
  loop do
  if the frontier is empty then return failure
  choose a leaf node and remove it from the frontier
  if the node contains a goal state then return the corresponding solution
  expand the chosen node, adding the resulting nodes to the frontier
\end{code}

\begin{code}[pseudo]{ This is a very very very very very very very very very very very very very very very very very very long title}
  function Tree-Search(problem) returns a solution, or failure
  initialize the frontier using the initial state of problem
  loop do
  if the frontier is empty then return failure
  choose a leaf node and remove it from the frontier
  if the node contains a goal state then return the corresponding solution
  expand the chosen node, adding the resulting nodes to the frontier

  function Tree-Search(problem) returns a solution, or failure
  initialize the frontier using the initial state of problem
  loop do
  if the frontier is empty then return failure
  choose a leaf node and remove it from the frontier
  if the node contains a goal state then return the corresponding solution
  expand the chosen node, adding the resulting nodes to the frontier
\end{code}

\section{Boxes}
\subsection{Infobox}
\begin{infobox}[noprefix=false]
  \lipsum[1-1]
\end{infobox}

\subsubsection{Infobox in a minipage}
\noindent
\begin{minipage}{0.5\linewidth}% <-- comment is important
  \begin{flushleft}
    \begin{infobox}[title={This is the title},noprefix,nofloat,width=0.95\linewidth]
      I am the left info box.
    \end{infobox}
  \end{flushleft}
\end{minipage}% <-- comment is important
\begin{minipage}{0.5\linewidth}% <-- comment is important
  \begin{flushright}
    \begin{infobox}[title={This is the title},noprefix,nofloat,width=0.95\linewidth]
      I am the right info box.
    \end{infobox}
  \end{flushright}
\end{minipage}% <-- comment is important

\subsection{Highlightbox}
\begin{highlightbox}
  \(1 + 2 = 3\)
\end{highlightbox}

\end{document}