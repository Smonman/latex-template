%%% TEMPLATE
% Author: Simon Josef Kreuzpointner
% Date: 06.10.2022

\documentclass[10pt,titlepage]{article}

% PACKAGES
\usepackage[a4paper,top=3cm,bottom=2cm,left=2cm,right=2cm,headsep=1cm]{geometry} % layout
\usepackage[english,ngerman]{babel} % language
\usepackage{iflang} % conditional statements based on selected language
\usepackage{ifthen} % for conditinal branching
\usepackage[skip=4pt, indent=10pt]{parskip} % spacing between paragraphs
\usepackage{titling} % title
\usepackage{fancyhdr} % header and footer
\usepackage{amsmath} % math
\usepackage{amssymb} % math symbols
\usepackage{mathtools} % enhances amsmath
\usepackage{microtype} %improves the spacing between words and letters
\usepackage{graphicx} % graphics, images
\usepackage{array} % extension of array and tabular environments
\usepackage{xcolor} % work with colors
\usepackage{fp} % fixed point arithmetic
\usepackage{booktabs} % better table design
\usepackage{tabularx} % better table layout
\usepackage{makecell} % table cell configuration
\usepackage{listings} % code listing
\usepackage{pgfkeys} % pgf keys
\usepackage{tikz} % draw diagrams and figures
\usepackage{varwidth} % variable width mini-page
\usepackage{float} % 
\usepackage[most]{tcolorbox} % for infoboxes
\usepackage[utf8]{inputenc} % input encoding
\usepackage[T1]{fontenc} % font encoding for umlaute
\usepackage{lipsum} % dummy text
% MUST BE PLACED LAST
\usepackage[colorlinks=true, allcolors=blue]{hyperref}

%%% COLORS
% standard colors
\definecolor{myyellow}{rgb}{0.94, 0.82, 0.007}
\definecolor{myred}{rgb}{0.75, 0.16, 0.18}
\definecolor{myblue}{rgb}{0.13, 0.34, 0.53}
\definecolor{mydarkblue}{rgb}{0.1, 0.16, 0.27}
\definecolor{mylightblue}{rgb}{0.39, 0.68, 1}
\definecolor{mydarkgreen}{rgb}{0, 0.4, 0}
\definecolor{mysmoke}{rgb}{0.9, 0.9, 0.9}
% lstlisting
\definecolor{codegray}{HTML}{707070}
\definecolor{codelightgray}{HTML}{FAFAFA}
\definecolor{codegreen}{HTML}{24751B}
\definecolor{codeblue}{HTML}{1733BF}

%%% GRAPHIXS PATH
\graphicspath{{../img/}}

%%% TITLE STYLE
\pretitle{
  \begin{center}
    \Huge
    }
    \posttitle{
  \end{center}
  \noindent\vrule height 2.5pt width \textwidth
  \vskip .75em plus .25em minus .25em
}
\preauthor{
  \begin{center}
    \itshape---
    }
    \postauthor{
    ---
  \end{center}
}

%%% PAGESTYLE
\pagestyle{fancy}
\fancyhf{}
\fancyhead[L]{\nouppercase{\leftmark}}
\renewcommand{\headrulewidth}{0.2pt}
\fancyfoot[C]{\thepage}
\renewcommand{\footrulewidth}{0.2pt}
\renewcommand{\footruleskip}{1em}
\renewcommand{\subsectionmark}[1]{\markright{\thesubsection\ #1}}

%%% ITEMIZE STYLE
\renewcommand\labelitemi{$\vcenter{\hbox{\small$\bullet$}}$}

%%% LSTLISTING STYLE
\lstdefinestyle{lststandardstyle}{
  backgroundcolor=\color{codelightgray},
  basicstyle=\footnotesize\ttfamily,
  breakatwhitespace=false,
  breaklines=true,
  captionpos=t,
  commentstyle=\color{codegray},
  deletekeywords={},
  extendedchars=true,
  firstnumber=1,
  frame=single,
  keepspaces=true,
  keywordstyle=\color{codeblue}\bfseries,
  morekeywords={*,...},
  numbers=left,
  numbersep=5pt,
  numberstyle=\tiny\color{codegray},
  rulecolor=\color{black},
  showspaces=false,
  showstringspaces=false,
  showtabs=false,
  stepnumber=1,
  stringstyle=\color{codegreen},
  tabsize=4,
  escapeinside={(*}{*)},
  linewidth=\columnwidth,
  xleftmargin=3.4pt,
  xrightmargin=3.4pt,
}
\lstdefinestyle{lstoutput}{
  backgroundcolor=\color{codelightgray},
  basicstyle=\footnotesize\ttfamily,
  breakatwhitespace=false,
  breaklines=true,
  captionpos=t,
  extendedchars=true,
  firstnumber=1,
  frame=single,
  keepspaces=true,
  rulecolor=\color{black},
  stepnumber=0,
  tabsize=4,
  title=Output,
  escapeinside={(*}{*)},
}
\lstdefinelanguage{pseudo}{
  mathescape=true,
  basicstyle=\scriptsize,
  keywordstyle=\color{black}\bfseries\em,
  morekeywords={function, returns, return, if, then, else, and, or, not, loop, do, while, for, each, switch, case, foreach, begin, end, endif, until, repeat, null},
  sensitive=false,
  morecomment=[l]{//},
  morecomment=[s]{/*}{*/},
  morestring=[b]"
}
\lstset{
  style=lststandardstyle,
  framerule=0.8pt,
  rulesep=0pt
}

%%% TIKZ
% tikz libraries
\usetikzlibrary{
  shapes.geometric,
  arrows,
  positioning,
  bending,
  fit
}

% default node distance
\tikzset{node distance=1cm}

\tikzstyle{container} = [
rectangle,
very thick,
minimum width=4em,
minimum height=4em,
inner sep=1em,
draw=black,
fill=mysmoke,
label={\huge{Agent}}
]

\tikzstyle{roundedcontainer} = [
rectangle,
very thick,
rounded corners=1em,
minimum width=4em,
minimum height=4em,
inner sep=1em,
draw=black,
fill=smoke,
label={\huge{Agent}}
]

\tikzstyle{basic} = [
rectangle,
thick,
minimum width=7em,
minimum height=2em,
text centered,
draw=black,
fill=white,
execute at begin node={\begin{varwidth}{12em}},
      execute at end node={\end{varwidth}}
]

\tikzstyle{roundedbasic} = [
rectangle,
thick,
rounded corners,
minimum width=7em,
minimum height=2em,
text centered,
draw=black,
fill=white,
execute at begin node={\begin{varwidth}{12em}},
      execute at end node={\end{varwidth}}
]

\tikzstyle{arrow} = [
very thick,
->,
-stealth,
rounded corners=3mm,
shorten <=2pt,
shorten >=2pt
]

%%% PGF LAYERS
% declare a new background layer beneath the main layer
\pgfdeclarelayer{background}
\pgfsetlayers{background,main}

%%% TABULARX
% declare a new column type, that auto sizes but has ragged right text
\newcolumntype{Z}{>{\raggedright\let\newline\\\arraybackslash\hspace{0pt}}X}

%%% TABLE HEADER
% set table header to bold
\renewcommand\theadfont{\bfseries}
\renewcommand{\theadalign}{cl}
\renewcommand{\cellalign}{tl}

%%% TABLE ROWS
% table row spacing
\renewcommand{\arraystretch}{1.5}

%%% NEW COMMANDS
% math set
\newcommand{\mset}[1]{\ensuremath{\left\{\: #1 \:\right\}}}
% german qote
\newcommand{\gqote}[1]{\glqq #1\grqq}
% logical true and false
\newcommand{\ltrue}{\top}
\newcommand{\lfalse}{\bot}
% abs and norm
\DeclarePairedDelimiter\abs{\lvert}{\rvert}%
\DeclarePairedDelimiter\norm{\lVert}{\rVert}%
\makeatletter
\let\oldabs\abs
\def\abs{\@ifstar{\oldabs}{\oldabs*}}
\let\oldnorm\norm
\def\norm{\@ifstar{\oldnorm}{\oldnorm*}}
\makeatother

%%% CONDITIONS
\newenvironment{conditions}
{\par\noindent
\tabularx{\columnwidth}{>{$}l<{$} @{}>{${}}c<{{}$}@{} >{\raggedright\arraybackslash}X}}
{\endtabularx\par\vspace{\belowdisplayskip}}

%%% TCOLORBOX
\tcbset{enhanced}
\tcbuselibrary{listings}

%%% CODE BOX
\newtcblisting{code}[2][]{
  enhanced,
  boxsep=1pt,
  boxrule=0.8pt,
  toprule=5pt,
  titlerule=5pt,
  width=\columnwidth,
  adjusted title=\sffamily\bfseries #2,
  after upper={\par\hfill\textit{#1}},
  breakable,
  sharp corners,
  arc is angular,
  arc=10pt,
  rounded corners=northwest,
  colframe=black,
  colback=codelightgray,
  listing only,
  listing options={
      language=#1,
      frame=none,
    },
}

%%% INFO BOX
\newif\ifinfoboxNoPrefix
\newif\ifinfoboxNoFloat
\pgfkeys{
  /infobox/.is family, /infobox,
  title/.estore in=\infoboxTitle,
  noprefix/.is if=infoboxNoPrefix,
  nofloat/.is if=infoboxNoFloat,
  width/.estore in=\infoboxWidth,
  default/.style={title={},noprefix=true,nofloat=false,width=\linewidth}
}

\makeatletter
\newfloat{info@box}{!htbp}{loi}[section]
\makeatother
\floatname{info@box}{Infobox}

\newenvironment{infobox}[1][]{
  \pgfkeys{/infobox, default, #1}
  \def\infoboxTitlePrefixValue{\ifinfoboxNoPrefix \else \IfLanguageName{ngerman}{Wichtig: }{Info: } \fi}
  \def\infoboxTitleValue{\infoboxTitlePrefixValue \infoboxTitle}

  \ifinfoboxNoFloat \else \begin{info@box} \fi
      \begin{tcolorbox}[
          colback=mylightblue!10!white,
          colframe=myblue!80!black,
          adjusted title=\sffamily\bfseries \infoboxTitleValue,
          arc is angular,
          arc=5pt,
          sharp corners,
          rounded corners=southeast,
          boxrule=0.8pt,
          width=\infoboxWidth,
        ]
        }{
      \end{tcolorbox}
      \ifinfoboxNoFloat \else \end{info@box} \fi
}

%%% HIGHLIGHT BOX
\makeatletter
\newfloat{highlight@box}{!htbp}{loi}[section]
\makeatother
\floatname{highlight@box}{Highlightbox}

\newenvironment{highlightbox}{
  \begin{highlight@box}
    \begin{center}
      \begin{tcolorbox}[
          colback=white,
          colframe=black,
          notitle,
          arc=0pt,
          sharpish corners,
          boxrule=0.8pt,
          halign=flush center,
          width=0.4\columnwidth,
          lifted shadow={1mm}{-2mm}{3mm}{0.2mm}{black!50!white}
        ]
        \large
        }{
      \end{tcolorbox}
    \end{center}
  \end{highlight@box}
}

%%%
%%% PLACE CUSTOM STYLES HERE
%%%

%%%

\title{Title}
\author{Author}

\begin{document}
\maketitle

\tableofcontents
\newpage

\section{Introduction}
\lipsum[1-1]

\subsection{Subsection}
\lipsum[2-3]

\subsubsection{Subsubsection}
\lipsum[2-3]

\paragraph{Paragraph}
\lipsum[1-2]

\subparagraph{Subparagraph}
\lipsum[1-2]

\section{Itemize and Enumerate}
\subsection{Itemize}
\begin{itemize}
  \item I am an item
  \item I am an item
  \item I am an item
        \begin{itemize}
          \item I am an item
          \item I am an item
          \item I am an item
                \begin{itemize}
                  \item I am an item
                  \item I am an item
                  \item I am an item
                \end{itemize}
        \end{itemize}
\end{itemize}

\subsection{Enumerate}
\begin{enumerate}
  \item I am an item
  \item I am an item
  \item I am an item
        \begin{enumerate}
          \item I am an item
          \item I am an item
          \item I am an item
                \begin{enumerate}
                  \item I am an item
                  \item I am an item
                  \item I am an item
                \end{enumerate}
        \end{enumerate}
\end{enumerate}

\section{Math}

\subsection{Abs and Norm}
Commands \verb|\abs| and \verb|\norm|\footnote{see https://tex.stackexchange.com/a/43009} produce the following:
\begin{table}[!ht]
  \centering
  \begin{tabular}{ccc}
    \toprule
    \thead{command} & \thead{non-starred}       & \thead{starred}            \\
    \midrule
    \verb|\abs|     & \(\abs{\frac{1}{2}x^2}\)  & \(\abs*{\frac{1}{2}x^2}\)  \\
    \verb|\norm|    & \(\norm{\frac{1}{2}x^2}\) & \(\norm*{\frac{1}{2}x^2}\) \\
    \bottomrule
  \end{tabular}
\end{table}

The difference between \textit{starred} and \textit{non-starred} lies in the scaling of the bars.

\subsection{Set}
Command: \verb|\mset| produces the following:
\[\mset{1, 2, 3, 4}\]
\[\mset{a \mid \frac{a}{2} > 5}\]

\subsection{Conditions}
Environment: \verb|\begin{conditions}|\footnote{see https://tex.stackexchange.com/a/95842} can be used for the following:
\\
Boltzmann distribution: state occupation probability of a thermodynamical system within fixed temperature \(T\): \[p(x) = \alpha \cdot e^{-\frac{E(x)}{k \cdot T}}\] where:
\begin{conditions}
  x & \dots & state \\
  \alpha & \dots & degeneracy (= number of states \(x'\) with the same energy as \(x\)) \\
  E(x) & \dots & energy \\
  k & \dots & Bolzmann constant
\end{conditions}
It is possible to have different symbols instead of the dots.

\section{lstlistings}

\subsection{Python}
\begin{lstlisting}[language=python]
import numpy as np
    
def incmatrix(genl1,genl2):
    m = len(genl1)
    n = len(genl2)
    M = None #to become the incidence matrix
    VT = np.zeros((n*m,1), int)  #dummy variable
    
    #compute the bitwise xor matrix
    M1 = bitxormatrix(genl1)
    M2 = np.triu(bitxormatrix(genl2),1) 

    for i in range(m-1):
        for j in range(i+1, m):
            [r,c] = np.where(M2 == M1[i,j])
            for k in range(len(r)):
                VT[(i)*n + r[k]] = 1;
                VT[(i)*n + c[k]] = 1;
                VT[(j)*n + r[k]] = 1;
                VT[(j)*n + c[k]] = 1;
                
                if M is None:
                    M = np.copy(VT)
                else:
                    M = np.concatenate((M, VT), 1)
                
                VT = np.zeros((n*m,1), int)
    
    return M
\end{lstlisting}

\subsection{C++}

\begin{lstlisting}[language=c++]
#include <iostream>
using namespace std;

int main() {
    int n, t1 = 0, t2 = 1, nextTerm = 0;

    cout << "Enter the number of terms: ";
    cin >> n;

    cout << "Fibonacci Series: ";

    for (int i = 1; i <= n; ++i) {
        // Prints the first two terms.
        if(i == 1) {
            cout << t1 << ", ";
            continue;
        }
        if(i == 2) {
            cout << t2 << ", ";
            continue;
        }
        nextTerm = t1 + t2;
        t1 = t2;
        t2 = nextTerm;
        
        cout << nextTerm << ", ";
    }
    return 0;
}
\end{lstlisting}

\subsection{Output}

\begin{lstlisting}[style=lstoutput]
Enter a positive integer: 100
Fibonacci Series: 0, 1, 1, 2, 3, 5, 8, 13, 21, 34, 55, 89, 
\end{lstlisting}

\subsection{Pseudocode}

\begin{lstlisting}[language=pseudo]
function Tree-Search(problem) returns a solution, or failure
  initialize the frontier using the initial state of problem
  loop do
    if the frontier is empty then return failure
    choose a leaf node and remove it from the frontier
    if the node contains a goal state then return the corresponding solution
    expand the chosen node, adding the resulting nodes to the frontier
\end{lstlisting}

\begin{code}[pseudo]{code.pseudo}
  function Tree-Search(problem) returns a solution, or failure
  initialize the frontier using the initial state of problem
  loop do
  if the frontier is empty then return failure
  choose a leaf node and remove it from the frontier
  if the node contains a goal state then return the corresponding solution
  expand the chosen node, adding the resulting nodes to the frontier
\end{code}

\begin{code}[pseudo]{ This is a very very very very very very very very very very very very very very very very very very long title}
  function Tree-Search(problem) returns a solution, or failure
  initialize the frontier using the initial state of problem
  loop do
  if the frontier is empty then return failure
  choose a leaf node and remove it from the frontier
  if the node contains a goal state then return the corresponding solution
  expand the chosen node, adding the resulting nodes to the frontier

  function Tree-Search(problem) returns a solution, or failure
  initialize the frontier using the initial state of problem
  loop do
  if the frontier is empty then return failure
  choose a leaf node and remove it from the frontier
  if the node contains a goal state then return the corresponding solution
  expand the chosen node, adding the resulting nodes to the frontier
\end{code}

\section{Boxes}
\subsection{Infobox}
\begin{infobox}[noprefix=false]
  \lipsum[1-1]
\end{infobox}

\subsubsection{Infobox in a minipage}
\noindent
\begin{minipage}{0.5\linewidth}% <-- comment is important
  \begin{flushleft}
    \begin{infobox}[title={This is the title},noprefix,nofloat,width=0.95\linewidth]
      I am the left info box.
    \end{infobox}
  \end{flushleft}
\end{minipage}% <-- comment is important
\begin{minipage}{0.5\linewidth}% <-- comment is important
  \begin{flushright}
    \begin{infobox}[title={This is the title},noprefix,nofloat,width=0.95\linewidth]
      I am the right info box.
    \end{infobox}
  \end{flushright}
\end{minipage}% <-- comment is important

\subsection{Highlightbox}
\begin{highlightbox}
  \(1 + 2 = 3\)
\end{highlightbox}

\end{document}